\documentclass[10pt]{beamer}

\usepackage{fontspec}
\setmainfont{Roboto Mono}[]
\setsansfont{Roboto Mono}[]
\setmonofont{Roboto Mono}[]

\usepackage{graphicx}
\graphicspath{ {../img/} }

\usepackage[absolute,overlay]{textpos} % [showboxes]

\beamertemplatenavigationsymbolsempty

%% \lstset{
%%   language=ML,
%%   keywordstyle=\color{blue},
%%   backgroundcolor=\color{lightgray}
%% }

\title{Работа с процессами на низком уровне}

\begin{document}

\begin{frame}
  \frametitle{Многопоточность}
  \begin{itemize}
    \item главная фишка BEAM,
    \item основа масштабируемости, 
    \item распределённости,
    \item устойчивости к ошибкам.
  \end{itemize}
\end{frame}

\begin{frame}
  \frametitle{Процессы и потоки}
  Три разные сущности: 
  \begin{itemize}
    \item процесс (process), 
    \item поток (thread) 
    \item и зеленый поток (green thread).
  \end{itemize}
\end{frame}

\begin{frame}
  \frametitle{Процессы BEAM}
  \begin{itemize}
    \item имеют свою память (стек и кучу),
    \item владеют ресурсами (сокетами),
    \item не блокируют друг друга,
    \item легковесные.
  \end{itemize}
\end{frame}

\begin{frame}
  \frametitle{Процессы BEAM}
  \begin{itemize}
    \item стартуют за 3-5 микросекунд,
    \item занимают 2.5 Кб памяти,
    \item лимит по умолчанию $2^{18}$ (262 тысячи),
    \item максимум $2^{27}$ (134 миллиона).
  \end{itemize}
\end{frame}

\begin{frame}
  \frametitle{Типичная область применения BEAM}
  Сервера, которые должны обслуживать большое количество клиентов. 
  \par \bigskip
  Особенно если соединения с клиентами являются долгоживущими.
\end{frame}

\end{document}
